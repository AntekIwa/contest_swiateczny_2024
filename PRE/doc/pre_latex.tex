\documentclass[a4paper,11pt]{article}
\usepackage[utf8]{inputenc}
\usepackage[T1]{fontenc}
\usepackage[polish]{babel}
\usepackage{amsmath}
\usepackage{amssymb}
\usepackage{geometry}
\usepackage{graphicx}
\usepackage{multicol}
\geometry{top=0.5in, bottom=0.8in, left=0.8in, right=0.8in}

% Customizable parameters
% Wstaw pełny tytuł zadania
\newcommand{\tasktitle}{Świąteczne Prezenty}
% Wstaw skrócony tytuł zadania
\newcommand{\taskshort}{PRE}
% Wstaw informacje o konkursie
\newcommand{\contestinfo}{Konkurs Świąteczny 2024 - Grupa Początkująca.}
% Wstaw limit pamięci
\newcommand{\memorylimit}{10 MB}
% Wstaw dane przykładowe wejścia
\newcommand{\exampleinput}{3 2\\1 4 7\\+ 5\\?}
% Wstaw dane przykładowe wyjścia
\newcommand{\exampleoutput}{6\\6}
% Wstaw wyjaśnienie przykładu
\newcommand{\explanation}{Początkowy stos to [1, 4, 7]. Po operacji + 5, stos staje się [1, 4, 7, 5]. Operacja ? zwraca różnicę między maksymalną a minimalną wartością: $7 - 1 = 6$.}
% Wstaw tabelę podzadań
\newcommand{\subtasktable}{% 
\begin{tabular}{|c|c|c|}
\hline
Podzadanie & Ograniczenia & Punkty \\
\hline
1 & $q, n \leq 10$ & 10 \\
2 & $q, n \leq 1000$ & 30 \\
3 & $q, n \leq 100000$ & 60 \\
\hline
\end{tabular}}

\begin{document}

% Nagłówek zadania
\noindent\textbf{\LARGE Zadanie: \taskshort} \\
\textbf{\Large \tasktitle} \\
\rule{\textwidth}{0.4pt} \\
\small \contestinfo \
\textbf{Dostępna pamięć: \memorylimit.}

% Wstaw obrazek
\begin{center}
\includegraphics[width=0.6\textwidth]{zdj.png} % Zmień "zdj.png" na nazwę pliku z obrazkiem
\end{center}

% Treść zadania
\noindent Bajtek, robiąc przerwę w świątecznych porządkach, zajrzał pod choinkę i zauważył stos prezentów o różnych wartościach. Z czasem zaczęły pojawiać się nowe paczki, więc postanowił na bieżąco analizować ich stan.

Każdy prezent miał wartość liczbową, a Bajtek chciał wiedzieć dwie rzeczy:
\begin{itemize}
    \item Jak zmieni się stos, gdy pojawi się nowy prezent.
    \item Jaki jest poziom niesprawiedliwości, czyli różnica między największym a najmniejszym prezentem.
\end{itemize}
Prezentów przybywało coraz więcej, a Bajtek potrzebuje pomocy, by nadążyć za zmianami. Pomóż mu obsłużyć $q$ zapytań, uwzględniając początkowy stan stosu.

% Sekcja wejścia
\section*{Wejście}
Pierwszy wiersz wejścia zawiera dwie liczby całkowite $n$ i $q$ ($1 \leq n, q \leq 100\,000$) oznaczające początkową liczbę prezentów w stosie oraz liczbę zapytań.

Drugi wiersz zawiera $n$ liczb całkowitych $a_1, a_2, \dots, a_n$ ($1 \leq a_i \leq 10^9$) – początkowe wartości prezentów w stosie.

Dodatkowo, na początku należy wypisać poziom niesprawiedliwości dla początkowego stanu stosu.

Kolejne $q$ wierszy opisują zapytania w jednej z dwóch form:
\begin{itemize}
    \item \texttt{+ x} – dodaj nowy prezent o wartości $x$ ($1 \leq x \leq 10^9$),
    \item \texttt{?} – wypisz różnicę między największą a najmniejszą wartością w stosie.
\end{itemize}

% Sekcja wyjścia
\section*{Wyjście}
Dla każdego zapytania typu \texttt{?} wypisz w osobnym wierszu jedno całkowite liczba – różnicę między największą a najmniejszą wartością w stosie.
\newpage
% Sekcja przykładów
\section*{Przykład}
\begin{multicols}{2}
\noindent\textbf{Wejście:} \\
\exampleinput

\columnbreak

\noindent\textbf{Wyjście:} \\
\exampleoutput
\end{multicols}

\noindent Wyjaśnienie: \explanation

% Sekcja oceniania
\section*{Ocenianie}
Zestaw testów dzieli się na następujące podzadania:
\begin{center}
\subtasktable
\end{center}

% Stopka
\vspace*{\fill}
\noindent\rule{\textwidth}{0.4pt} \\
\small \textbf{Autor:} Antoni Iwanowski \hfill \textbf{VIII LO im. Władysława IV w Warszawie}

% Komentarze:
% W polu \tasktitle wpisz pełny tytuł zadania.
% W polu \taskshort wpisz skrócony tytuł zadania.
% W polu \contestinfo wpisz informacje o konkursie.
% W polu \memorylimit wpisz limit pamięci.
% W polach \exampleinput i \exampleoutput wpisz dane przykładowe wejścia i wyjścia.
% W polu \explanation wpisz wyjaśnienie przykładu.
% W polu \subtasktable wpisz tabelę z podzadaniami, ich ograniczeniami i punktami.
% W stopce wstaw autora i nazwę szkoły.
\end{document}
