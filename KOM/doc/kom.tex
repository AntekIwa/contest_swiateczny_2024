\documentclass[a4paper,11pt]{article}
\usepackage[utf8]{inputenc}
\usepackage[T1]{fontenc}
\usepackage[polish]{babel}
\usepackage{amsmath}
\usepackage{amssymb}
\usepackage{geometry}
\usepackage{graphicx}
\usepackage{multicol}
\geometry{top=0.5in, bottom=0.8in, left=0.8in, right=0.8in}

% Customizable parameters
% Wstaw pełny tytuł zadania
\newcommand{\tasktitle}{Kominy}
% Wstaw skrócony tytuł zadania
\newcommand{\taskshort}{KOM}
% Wstaw informacje o konkursie
\newcommand{\contestinfo}{Konkurs Świąteczny 2024 - Grupa Początkująca.}
% Wstaw limit pamięci
\newcommand{\memorylimit}{512 MB}
% Wstaw dane przykładowe wejścia
\newcommand{\exampleinput}{\\4 4 3
\\1 2 0 4
\\2 -1 -1 5
\\3 2 0 6
\\4 5 6 7}
% Wstaw dane przykładowe wyjścia
\newcommand{\exampleoutput}{\\4}
% Wstaw wyjaśnienie przykładu
\newcommand{\explanation}{Mikołaj zaczyna na polu \( g_{1,1} = 1 \). Następnie przemieszcza się, wykorzystując linę o maksymalnej długości 4, aby dotrzeć do mety przy \( g_{4,4} = 7 \).}
% Wstaw imię i nazwisko autora
\newcommand{\authorinfo}{Antoni Iwanowski}
% Wstaw nazwę szkoły
\newcommand{\schoolinfo}{VIII LO im. Władysława IV w Warszawie}
% Wstaw tabelę podzadań
\newcommand{\subtasktable}{% 
\begin{tabular}{|c|c|c|}
\hline
Podzadanie & Ograniczenia & Punkty \\
\hline
1 & $n, m \leq 10$ & 5 \\
2 & $n, m \leq 100$ & 35 \\
3 & $n, m \leq 800$ & 60 \\
\hline
\end{tabular}}

\begin{document}

% Nag\'lowek zadania
\noindent\textbf{\LARGE Zadanie: \taskshort} \\
\textbf{\Large \tasktitle} \\
\rule{\textwidth}{0.4pt} \\
\small \contestinfo \textbf{ Dostępna pamięć: \memorylimit.}

% Dodanie obrazu
\begin{center}
\includegraphics[width=0.6\textwidth]{zdj.jpg}
\end{center}

% Tre\'s\'c zadania
\section*{Tre\'s\'c zadania}
\noindent\normalsize
Święty Mikołaj wyrusza w swoją świąteczną podróż po Laponii, by dostarczyć prezenty dzieciom. Jego trasa prowadzi przez siatkę \( n \times m \), gdzie każda komórka reprezentuje komin o określonej trudności \( g_{i,j} \).

Mikołaj musi dotrzeć z pola \((1, 1)\) (start) na pole \((n, m)\) (meta), wybierając odpowiednią trasę i odpowiednio zarządzając swoją energią. W trakcie podróży może natrafić na różne sytuacje: jeśli przemieszcza się na pole o większej wysokości, musi użyć liny, której maksymalna długość \( M \) ogranicza różnicę wysokości, jaką może pokonać. Za każdy taki manewr zużywa 1 jednostkę energii. Gdy Mikołaj trafi na pole niższe lub równe wysokości, nie traci energii. Napotkane pola z magicznym napojem przywracają jego energię do pełnej wartości \( k \), natomiast pola oznaczone wartością \( -1 \) są niedostępne i nie może przez nie przejść. Mikołaj może poruszać się jedynie w prawo lub w dół.

Jeżeli nie da się dotrzeć do mety, należy wypisać słowo \texttt{NIE}.

\section*{Wejście}
Na wejściu znajdują się trzy liczby całkowite \( n, m, k \) ($1 \leq n, m \leq 800$, \( 1 \leq k \leq 10^4 \)), oznaczające wymiary planszy oraz maksymalną liczbę jednostek energii. Następnie podana jest siatka \( g_{i,j} \) o wymiarach \( n \times m \). Wartości \( g_{i,j} \) mogą być dodatnie (wysokość komina), równe 0 (pole zawierające magiczny napój) lub równe \( -1 \) (pole niedostępne).

\section*{Wyjście}
Na wyjściu należy wypisać jedną liczbę całkowitą — minimalną długość liny \( M \), która pozwala Mikołajowi dotrzeć z \( (1, 1) \) na \( (n, m) \). Jeśli nie da się dotrzeć do mety, należy wypisać słowo \texttt{NIE}.
\newpage
\section*{Przykład}
\vspace{-0.5cm}
\begin{multicols}{2}
\noindent\textbf{Wejście:}
\exampleinput

\columnbreak

\noindent\textbf{Wyjście:}
\exampleoutput
\end{multicols}

\noindent Wyjaśnienie: \explanation

\section*{Ocenianie}
Zestaw testów dzieli się na następujące podzadania:
\begin{center}
\subtasktable
\end{center}

% Stopka
\vspace*{\fill}
\noindent \rule{\textwidth}{0.4pt} \\
\textbf{Autor:} \authorinfo  \hfill \textbf{\schoolinfo} \\
\end{document}
