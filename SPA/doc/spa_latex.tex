\documentclass[a4paper,11pt]{article}
\usepackage[utf8]{inputenc}
\usepackage[T1]{fontenc}
\usepackage[polish]{babel}
\usepackage{amsmath}
\usepackage{amssymb}
\usepackage{geometry}
\usepackage{graphicx}
\usepackage{multicol}
\geometry{top=0.5in, bottom=0.8in, left=0.8in, right=0.8in}

% Customizable parameters
% Wstaw pełny tytuł zadania
\newcommand{\tasktitle}{Świąteczne Paczkomaty}
% Wstaw skrócony tytuł zadania
\newcommand{\taskshort}{SPA}
% Wstaw informacje o konkursie
\newcommand{\contestinfo}{Konkurs Świąteczny 2024 - Grupa Początkująca.}
% Wstaw limit pamięci
\newcommand{\memorylimit}{128 MB}
% Wstaw dane przykładowe wejścia
\newcommand{\exampleinput}{5 2}
% Wstaw dane przykładowe wyjścia
\newcommand{\exampleoutput}{8}
% Wstaw wyjaśnienie przykładu
\newcommand{\explanation}{Możliwymi kombinacjami są m.in. $1+1+1+1+1$, $2+1+1+1$, $1+2+1+1$, itd.}
% Wstaw imię i nazwisko autora
\newcommand{\authorinfo}{Antoni Iwanowski}
% Wstaw nazwę szkoły
\newcommand{\schoolinfo}{VIII LO im. Władysława IV w Warszawie}
% Wstaw tabelę podzadań
\newcommand{\subtasktable}{% 
\begin{tabular}{|c|c|c|}
\hline
Podzadanie & Ograniczenia & Punkty \\
\hline
1 & $n, k \leq 30$ & 5 \\
2 & $n, k \leq 1000$ & 35 \\
3 & $n, k \leq 10^6$ & 60 \\
\hline
\end{tabular}}

\begin{document}

% Nagłówek zadania
\noindent\textbf{\LARGE Zadanie: \taskshort} \\
\textbf{\Large \tasktitle} \\
\rule{\textwidth}{0.4pt} \\
\small \contestinfo \textbf{ Dostępna pamięć: \memorylimit.}

% Wstaw obrazek
\begin{center}
\includegraphics[width=0.8\textwidth]{zdj.png} % Zmień "zdj.png" na nazwę pliku z obrazkiem
\end{center}

% Treść zadania
\noindent Święty Mikołaj przygotowuje się do świątecznej nocy, a paczkomaty w całej Laponii mają kluczową rolę w dystrybucji prezentów. Każda skrytka w paczkomacie ma dokładnie wymiary $1 \times n$. Zadaniem elfów jest policzenie, na ile sposobów można zapakować prezenty do skrytki, używając pudełek o różnych długościach: $1, 2, \dots, k$. Ważne jest, aby całkowita długość pudełek idealnie wypełniła skrytkę. Ponieważ elfy nie umieją zapamiętać więcej niż 9 cyfr, wynik należy podać modulo $10^9 + 7$.

% Sekcja wejścia
\section*{Wejście}
W pierwszym wierszu wejścia znajdują się dwie liczby całkowite $n$ i $k$ ($1 \leq n, k \leq 10^6$), oznaczające odpowiednio długość skrytki paczkomatu oraz maksymalną długość pudełek.

% Sekcja wyjścia
\section*{Wyjście}
Na wyjściu należy wypisać jedną liczbę całkowitą, oznaczającą liczbę sposobów na zapakowanie prezentów w skrytce o długości $n$ modulo $10^9 + 7$. 

% Sekcja przykładów
\section*{Przykład}
\begin{multicols}{2}
\noindent\textbf{Wejście:}
\exampleinput

\noindent\textbf{Wyjście:}
\exampleoutput
\end{multicols}

\noindent Wyjaśnienie: \explanation

% Sekcja oceniania
\section*{Ocenianie}
Zestaw testów dzieli się na następujące podzadania:
\begin{center}
\subtasktable
\end{center}

% Stopka
\vspace{1cm}
\noindent \rule{\textwidth}{0.4pt} \\
\textbf{Autor:} \authorinfo  \hfill \textbf{\schoolinfo} \\
\end{document}
